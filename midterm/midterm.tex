\documentclass[12pt]{article}
\usepackage[margin=1in]{geometry} 
\usepackage{amsmath,amsthm,amssymb,amsfonts}
\usepackage{enumerate,listings,graphicx,epstopdf,siunitx}
\usepackage{color}
\graphicspath{~/Documents/school/fall16/stat586/midterm}

\sloppy
\definecolor{lightgray}{gray}{0.5}
 
\newcommand{\N}{\mathbb{N}}
\newcommand{\Z}{\mathbb{Z}}
\newcommand{\normD}[3]{\frac{1}{\sqrt{2\pi #1^2}}exp(\frac{-( #2 - #3)^2}{2 #1^2})} 
\newenvironment{problem}[2][Problem]{\begin{trivlist}
\item[\hskip \labelsep {\bfseries #1}\hskip \labelsep {\bfseries #2.}]
  \vspace{1 cm}
}{\end{trivlist}}

\begin{document}
\title{Midterm}
\author{Taylor Bodin}
\maketitle

\section*{Problem 1}

\subsection*{a.}

\subsubsection*{Derive $f_Y(y)$ }
\begin{align*}
  g(x) &= X^3 \\
  g^{-1}(y) &= \sqrt[3]{y} \\
  g^{'}(x) &= 3x^2 \\
  g^{'}(g^{-1}(y)) &= 3y^{\frac{2}{3}} \\
  f_X(g^{-1}(y)) &= 42y^{\frac{5}{3}}(1-\sqrt[3]{y}) \\
  g(0) = 0 \\
  g(1) = 1 \\
  f_Y(y) &= 
  \begin{cases}
    \frac{f_X(g^{-1}(y))}{g^{'}(g^{-1}(y))} 
    = \frac{42y^{\frac{5}{3}}(1-\sqrt[3]{y})}{3y^{\frac{2}{3}}}
    = 14y(1-\sqrt[3]{y}), & 0 \leq y \leq 1 \\
    0, & otherwise
  \end{cases}
\end{align*}

\subsubsection*{Show that $\int_{-\infty}^\infty f_Y(y) = 1$}
\begin{align*}
  \int_0^1 14y-14y^\frac{4}{3}dy 
  &= 7y^2 - 14\left(\frac{3}{7}\right)y^{\frac{7}{3}}\big|_0^1 \\
  &= 7y^2 - 6y^{\frac{7}{3}}\big|_0^1 \\
  &= 7 - 6 - 0 = 1
\end{align*}

\subsection*{b.}

\subsubsection*{Derive $f_Y(y)$}
\begin{align*}
  g(x) &= 4X + 3 \\
  g^{-1}(y) &= \frac{y-3}{4} \\
  g^{'}(x) &= 4 \\
  g^{'}(g^{-1}(y)) &= 4 \\
  f_X(g^{-1}(y)) &= 7e^{-7\frac{y-3}{4}} \\
  g(0) = 3 \\
  g(\infty) = \infty \\
  f_Y(y) &= 
  \begin{cases}
    \frac{f_X(g^{-1}(y))}{g^{'}(g^{-1}(y))} 
    = \frac{7e^{-7\frac{y-3}{4}}}{4}, & 3 \leq y < \infty \\
    0, & otherwise
  \end{cases}
\end{align*}

\subsubsection*{Show that $\int_{-\infty}^\infty f_Y(y) = 1$}
\begin{align*}
  \int_3^\infty \frac{7e^{-7\frac{y-3}{4}}}{4}dy
  &= \frac{7}{4}e^{\frac{21}{4}}\int_3^\infty e^{\frac{-7}{4}y}dy \\
  &= \frac{7}{4}e^{\frac{21}{4}}\frac{-4}{7}\left[e^{\frac{-7}{4}y} \right]\big|_3^\infty \\
  &= -e^{\frac{21}{4}}\left[ e^{-\infty} - e^{\frac{-21}{4}}  \right] \\
  &= 1
\end{align*}

\subsection*{c.}

\subsubsection*{Derive $f_Y(y)$}
\begin{align*}
  g(x) &= X^2 \\
  g^{-1}(y) &= \sqrt{y} \\
  g^{'}(x) &= 2x \\
  g^{'}(g^{-1}(y)) &= 2\sqrt{y} \\
  f_X(g^{-1}(y)) &= 20\sqrt{y}^2(1-\sqrt{y})^2 = 30y - 60y^{\frac{3}{2}} + 30y^2 \\
  g(0) = 0 \\
  g(1) = 1 \\
  f_Y(y) &= 
  \begin{cases}
    \frac{f_X(g^{-1}(y))}{g^{'}(g^{-1}(y))} 
    = \frac{30y - 60y^{\frac{3}{2}} + 30y^2}{2\sqrt{y}}
    = 15\sqrt{y}(1-2\sqrt{y} + y) 
    = 15y^{\frac{1}{2}}-30y+15y^{\frac{3}{2}}, & 0 \leq y < 1 \\
    0, & otherwise
  \end{cases}
\end{align*}

\subsubsection*{Show that $\int_{-\infty}^\infty f_Y(y) = 1$} 
\begin{align*}
  \int_0^1 15y^{\frac{1}{2}}-30y+15y^{\frac{3}{2}}dy
  &= 10y^{\frac{3}{2}}-15y^2+6y^{\frac{5}{2}}\big|_0^1 \\
  &= 10-15+6-0 \\
  &= 1
\end{align*}

\section*{Problem 2}

\subsection*{a.}

\subsubsection*{Third Raw Moment}
\begin{align*}
  E[X^3] &= \sum_{x=0}^\infty \frac{x^3\lambda^x e^{-\lambda}}{x!} \\
  &= e^{-\lambda} \sum_{x=0}^{\infty} \frac{x^2\lambda^x}{(x-1)!} \\
  &= e^{-\lambda} \sum_{x=0}^{\infty} \frac{(m+1)^2\lambda^{m+1}}{m!} \\
  &= e^{-\lambda}\lambda \sum_{x=0}^{\infty} (m^2+2m+1)\frac{\lambda^m}{m!} \\
  &= e^{-\lambda}\lambda \left[\sum_{x=0}^{\infty} m^2\frac{\lambda^m}{m!}
    + 2\sum_{x=0}^{\infty} m\frac{\lambda^m}{m!}
    + \sum_{x=0}^{\infty} \frac{\lambda^m}{m!}\right] \\
  &= e^{-\lambda}\lambda \left[\sum_{x=0}^{\infty} m\frac{\lambda^{m-1}}{(m-1)!}
    + 2\lambda e^{\lambda}
    + e^{\lambda}\right] \\
  &= e^{-\lambda}\lambda \left[\lambda\sum_{x=0}^{\infty} (n+1)\frac{\lambda^{n}}{n!}
    + 2\lambda e^{\lambda}
    + e^{\lambda}\right] \\
  &= e^{-\lambda}\lambda \left[\lambda\left[ \lambda e^{\lambda} + e^{\lambda} \right]
    + 2\lambda e^{\lambda}
    + e^{\lambda}\right] \\
  &= e^{-\lambda}\lambda \left[\lambda^2 e^{\lambda} + 3\lambda e^{\lambda} + e^{\lambda} \right] \\
  &= \lambda^3 + 3\lambda^2 + 1
\end{align*}

\subsubsection*{Fourth Raw Moment}
\begin{align*}
  E[X^4] &= \sum_{x=0}^\infty \frac{x^4\lambda^x e^{-\lambda}}{x!} \\
  &= e^{-\lambda} \sum_{x=0}^{\infty} \frac{x^3\lambda^x}{(x-1)!} \\
  &= e^{-\lambda} \sum_{x=0}^{\infty} \frac{(m+1)^3\lambda^{m+1}}{m!} \\
  &= e^{-\lambda}\lambda \sum_{x=0}^{\infty} (m^3+3m^2+3m+1)\frac{\lambda^m}{m!} \\
  &= e^{-\lambda}\lambda \left[\sum_{x=0}^{\infty} m^3\frac{\lambda^m}{m!}
    + 3\sum_{x=0}^{\infty} m^2\frac{\lambda^m}{m!}
    + 3\sum_{x=0}^{\infty} m\frac{\lambda^m}{m!}
    + \sum_{x=0}^{\infty} \frac{\lambda^m}{m!}\right] \\
  &= e^{-\lambda}\lambda \left[\sum_{x=0}^{\infty} m^2 \frac{\lambda^{m-1}}{(m-1)!}
    + 3[\lambda^2e^{\lambda}+\lambda e^{\lambda}]
    + 3\lambda e^{\lambda}
    + e^{\lambda}\right] \\
  &= e^{-\lambda}\lambda \left[\lambda\sum_{x=0}^{\infty} (n+1)^2 \frac{\lambda^{n}}{n!}
    + 3[\lambda^2e^{\lambda}+\lambda e^{\lambda}]
    + 3\lambda e^{\lambda}
    + e^{\lambda}\right] \\
  &= e^{-\lambda}\lambda \left[\lambda\sum_{x=0}^{\infty} (n^2+2n+1) \frac{\lambda^{n}}{n!}
    + 3[\lambda^2e^{\lambda}+\lambda e^{\lambda}]
    + 3\lambda e^{\lambda}
    + e^{\lambda}\right] \\
  &= e^{-\lambda}\lambda \left[\lambda^3 e^{\lambda} + 3\lambda^2 e^{\lambda} + \lambda e^{\lambda}
    + 3\lambda^2e^{\lambda}+ 3\lambda e^{\lambda}
    + 3\lambda e^{\lambda}
    + e^{\lambda}\right] \\   
  &= e^{-\lambda}\lambda \left[\lambda^3 e^{\lambda} + 6\lambda^2 e^{\lambda} 
    + 7\lambda e^{\lambda} + e^{\lambda} \right] \\
  &= \lambda^4 + 6\lambda^3 + 7\lambda^2 + \lambda
\end{align*}


\subsection*{b.}

\subsubsection*{Third Raw Moment}
\begin{align*}
        E[X^3] &= \frac{1}{\Gamma(\alpha)\beta^\alpha} 
          \int_0^\infty x^3 x^{\alpha-1} e^{\frac{-x}{\beta}}dx \\
        &= \frac{1}{\Gamma(\alpha)\beta^\alpha} 
          \int_0^\infty x^{\alpha+2} e^{\frac{-x}{\beta}}dx \\
        &= \frac{1}{\Gamma(\alpha)\beta^\alpha} \frac{\Gamma(\alpha+3)\beta^{\alpha+3}}{1} 
          \int_0^\infty \frac{1}{\Gamma(\alpha+3)\beta^{\alpha+3}} x^{\alpha+2} e^{\frac{-x}{\beta}}dx & &
          \textrm{pdf integral evaluates to 1}\\
        &= \frac{1}{\Gamma(\alpha)\beta^\alpha} \frac{\Gamma(\alpha+3)\beta^{\alpha+3}}{1} \\
        &= (\alpha + 2)(\alpha + 1)\alpha \beta^3
\end{align*}

\subsubsection*{Fourth Raw Moment}
\begin{align*}
        E[X^4] &= \frac{1}{\Gamma(\alpha)\beta^\alpha} 
          \int_0^\infty x^4 x^{\alpha-1} e^{\frac{-x}{\beta}}dx \\
        &= \frac{1}{\Gamma(\alpha)\beta^\alpha} 
          \int_0^\infty x^{\alpha+3} e^{\frac{-x}{\beta}}dx \\
        &= \frac{1}{\Gamma(\alpha)\beta^\alpha} \frac{\Gamma(\alpha+4)\beta^{\alpha+4}}{1} 
          \int_0^\infty \frac{1}{\Gamma(\alpha+4)\beta^{\alpha+4}} x^{\alpha+3} e^{\frac{-x}{\beta}}dx & &
          \textrm{pdf integral evaluates to 1}\\
        &= \frac{1}{\Gamma(\alpha)\beta^\alpha} \frac{\Gamma(\alpha+4)\beta^{\alpha+4}}{1} \\
        &= (\alpha + 3)(\alpha + 2)(\alpha + 1)\alpha \beta^4
\end{align*}

\subsection*{c.}

\subsubsection*{Third Raw Moment}
\begin{align*}
        E[X^3] &= \frac{1}{B(\alpha,\beta)} \int_0^1 x^3 x^{\alpha-1} (1-x)^{\beta-1}dx \\
        &= \frac{1}{B(\alpha,\beta)} \int_0^1 x^{\alpha+2} (1-x)^{\beta-1}dx \\
        &= \frac{1}{B(\alpha,\beta)}\frac{B(\alpha+3,\beta)}{1}
          \int_0^1 \frac{1}{B(\alpha+3,\beta)} x^{\alpha+2} (1-x)^{\beta-1}dx & &
          \textrm{pdf integral evaluates to 1} \\
        &= \frac{B(\alpha+3,\beta)}{B(\alpha,\beta)} \\
        &= \frac{\Gamma(\alpha+3)\Gamma(\beta)}{\Gamma(\alpha+\beta+3)}
           \frac{\Gamma(\alpha+\beta)}{\Gamma(\alpha)\Gamma(\beta)} \\
        &= \frac{(\alpha+2)(\alpha+1)\alpha}{(\alpha+\beta+2)(\alpha+\beta+1)(\alpha+\beta)}
\end{align*}

\subsubsection*{Fourth Raw Moment}
\begin{align*}
        E[X^4] &= \frac{1}{B(\alpha,\beta)} \int_0^1 x^4 x^{\alpha-1} (1-x)^{\beta-1}dx \\
        &= \frac{1}{B(\alpha,\beta)} \int_0^1 x^{\alpha+3} (1-x)^{\beta-1}dx \\
        &= \frac{1}{B(\alpha,\beta)}\frac{B(\alpha+4,\beta)}{1}
          \int_0^1 \frac{1}{B(\alpha+4,\beta)} x^{\alpha+3} (1-x)^{\beta-1}dx & &
          \textrm{pdf integral evaluates to 1} \\
        &= \frac{B(\alpha+4,\beta)}{B(\alpha,\beta)} \\
        &= \frac{\Gamma(\alpha+4)\Gamma(\beta)}{\Gamma(\alpha+\beta+4)}
           \frac{\Gamma(\alpha+\beta)}{\Gamma(\alpha)\Gamma(\beta)} \\
        &= \frac{(\alpha+3)(\alpha+2)(\alpha+1)\alpha}
        {(\alpha+\beta+3)(\alpha+\beta+2)(\alpha+\beta+1)(\alpha+\beta)}
\end{align*}

\section*{Problem 3}

\subsection*{a.}

\subsubsection*{First Raw Moment, $\mu$}
\begin{align*}
  E[X] &= \sum_{x=0}^\infty x\binom{r+x-1}{x}p^r (1-p)^x \\
  &= p^r \sum_{x=0}^\infty x\binom{-r}{x}(-1)^x(1-p)^x \\
  &= p^r \sum_{x=0}^\infty x\frac{-r!}{x!(-r-x)!}(-1)^x(1-p)^x \\
  &= p^r \sum_{x=0}^\infty \frac{-r!}{(x-1)!(-r-x)!}(-1)^x(1-p)^x \\
  &= p^r \sum_{x=0}^\infty \frac{-r!}{(x-1)!(-r-x)!}(-1)^x(1-p)^x \\
  &= p^r(-1)(1-p) \sum_{x=0}^\infty \frac{-r!}{(x-1)!(-r-x)!}(-1)^{x-1}(1-p)^{x-1} \\
  &= p^r(-1)(1-p) \sum_{x=0}^\infty \frac{-r!}{m!(-r-m-1)!}(-1)^{m}(1-p)^{m} \\
  &= p^r(-1)(1-p)(-r) \sum_{x=0}^\infty \frac{(-r-1)!}{m!(-r-1-m)!}(-1)^{m}(1-p)^{m} \\
  &= p^r(-1)(1-p)(-r) \sum_{x=0}^\infty \binom{-r-1}{m}(-1)^{m}(1-p)^{m} \\
  &= p^r(-1)(1-p)(-r)(1-(1-p))^{-r-1} \\
  &= rp^r(1-p)p^{-r-1} \\
  &= r\frac{1-p}{p} \\
\end{align*}

\subsubsection*{Second Raw Moment}
\begin{align*}
  E[X] &= \sum_{x=0}^\infty x^2\binom{r+x-1}{x}p^r (1-p)^x \\
  &= p^r \sum_{x=0}^\infty x^2\binom{-r}{x}(-1)^x(1-p)^x \\
  &= p^r \sum_{x=0}^\infty x^2\frac{-r!}{x!(-r-x)!}(-1)^x(1-p)^x \\
  &= p^r \sum_{x=0}^\infty x\frac{-r!}{(x-1)!(-r-x)!}(-1)^x(1-p)^x \\
  &= p^r \sum_{x=0}^\infty x\frac{-r!}{(x-1)!(-r-x)!}(-1)^x(1-p)^x \\
  &= p^r(-1)(1-p) \sum_{x=0}^\infty x\frac{-r!}{(x-1)!(-r-x)!}(-1)^{x-1}(1-p)^{x-1} \\
  &= p^r(-1)(1-p) \sum_{x=0}^\infty (m+1)\frac{-r!}{m!(-r-m-1)!}(-1)^{m}(1-p)^{m} \\
  &= p^r(-1)(1-p)(-r) \sum_{x=0}^\infty (m+1)\frac{(-r-1)!}{m!(-r-1-m)!}(-1)^{m}(1-p)^{m} \\
  &= p^r(1-p)r \left[\sum_{x=0}^\infty m\frac{(-r-1)!}{m!(-r-1-m)!}(-1)^{m}(1-p)^{m}\right] \\
  &+ p^r(1-p)r \left[\sum_{x=0}^\infty \frac{(-r-1)!}{m!(-r-1-m)!}(-1)^{m}(1-p)^{m}\right]\\
  &= p^r(1-p)r \left[\sum_{x=0}^\infty m\frac{(-r-1)!}{m!(-r-1-m)!}(-1)^{m}(1-p)^{m}\right] \\
  &+ p^r(1-p)r \left[p^{-r-1}\right]\\
  &= p^r(1-p)r \left[\sum_{x=0}^\infty \frac{(-r-1)!}{(m-1)!(-r-1-m)!}(-1)^{m}(1-p)^{m}\right] 
  + r\frac{1-p}{p} \\
  &= p^r(1-p)r \left[(-1)(1-p)\sum_{x=0}^\infty \frac{(-r-1)!}{(n)!(-r-1-n-1)!}(-1)^{n}(1-p)^{n}\right] 
  + r\frac{1-p}{p} \\
  &= p^r(1-p)r \left[(r+1_(1-p)\sum_{x=0}^\infty \frac{(-r-2)!}{(n)!(-r-2-n)!}(-1)^{n}(1-p)^{n}\right] 
  + r\frac{1-p}{p} \\
  &= p^r(1-p)r \left[(r+1)(1-p)p^{-r-2}\right] + r\frac{1-p}{p} \\
  &= \frac{r(r+1)(1-p)^2}{p^2} + r\frac{1-p}{p}
\end{align*}

\subsection*{b.}
\begin{align*}
  E[e^{tX}] &= M_X(t) = \sum_0^\infty e^{tX} \binom{r+x-1}{x}p^r(1-p)^x \\
  &= p^r \sum_{x=0}^\infty e^{tX}\binom{-r}{x}(-1)^x(1-p)^x \\
  &= p^r \sum_{x=0}^\infty \binom{-r}{x}(-1)^x(e^{t}(1-p))^x \\
  &= p^r \sum_{x=0}^\infty \binom{-r}{x}(-1)^x(e^{t}(1-p))^x \\
  &= p^r (1-((e^{t}(1-p)))^-r \\
  &= \frac{p^r}{(1-((e^{t}(1-p)))^r} \\
  &= \frac{p^r}{(1-e^{t}(1-p))^r} \\
\end{align*}

\subsection*{c.}

\subsubsection*{First Moment}
\begin{align*}
  \frac{d}{dt}M_X(t) &= -r\frac{p^r}{(1-e^{t}(1-p))^{r+1}}(-(1-p))e^t, & & \textrm{chain rule} \\
  \frac{d}{dt}M_X(0) &= -r\frac{p^r}{(1-1+p))^{r+1}}(-(1-p)) \\
  &= -r\frac{p^r}{p^{r+1}}(-(1-p)) \\
  &= r\frac{1-p}{p^1} \\
  &= r\frac{1-p}{p}, & & \textrm{This matches the answer from part A}
\end{align*}

\subsubsection*{Second Moment}
\begin{align*}
  \frac{d}{dt}M_X(t) &= -r\frac{p^r}{(1-e^{t}(1-p))^{r+1}}(-(1-p))e^t, & & \textrm{chain rule} \\
  \frac{d^2}{dt^2}M_X(t) 
  &= \frac{d}{dt}\left[\frac{p^r}{(1-e^{t}(1-p))^{r+1}}\left(r(1-p)e^t\right) \right] \\
  &= (-r-1)\frac{p^r}{(1-e^{t}(1-p))^{r+2}}(-(1-p)e^t)\left(r(1-p)e^t\right) \\
  &+ \frac{p^r}{(1-e^{t}(1-p))^{r+1}}\left(r(1-p)e^t\right) \\
  \frac{d^2}{dt^2}M_X(0) &= r(r+1)(1-p)^2\frac{p^r}{p^{r+2}}
  + r(1-p)\frac{p^r}{p^{r+1}} \\
  &= \frac{r(r+1)(1-p)^2}{p^2} + r\frac{1-p}{p}, & & \textrm{This matches the answer from part A}
\end{align*}



\end{document} 
